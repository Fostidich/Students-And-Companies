\chapter{Installation guide}

Below are the steps to launch the system applications for local hosting.

\section{Install Docker Compose}

The installation guide for Docker Compose can be found at the following link.

\begin{center}
\url{https://docs.docker.com/compose/install}
\end{center}

\section{Download the configuration file}

Download the \verb|docker-compose.yml| file found in the root of the repository.
To do so, you can simply open a terminal and run the following command.

\begin{center}
\verb|curl -O https://raw.githubusercontent.com/Fostidich/\| \\
\verb|OstidichSalariRivitti-StudentsAndCompanies/main/docker-compose.yml|
\end{center}

\section{Start the application}

Stay in the folder containing the \verb|docker-compose.yml| file that you've just downloaded.

Depending on how Docker Compose has been installed, run the corresponding command: use

\begin{center}
\verb|docker compose up --pull always|
\end{center}

if you installed Docker Compose as a plugin; use

\begin{center}
\verb|docker-compose up --pull always|
\end{center}

if you installed it as a standalone binary.

\subsection{Troubleshoot}

Since Docker Compose can be installed in multiple ways, and since slight differences can be found depending on the running OS, below is a list of possible solutions to try if the previous command fails.

\begin{itemize}

    \item If you're using Docker Desktop, be sure that it has been previously launched.
    \item If on Windows, try to launch the terminal (CMD or PowerShell) with administrator rights.
    \item If on MacOS or Linux, try prepending a \verb|sudo| before the previous commands.
    \item It may happen that the ports that have been set in the \verb|docker-compose.yml| are already in use.
    If so, feel free to change them, but make sure they still match accordingly.

\end{itemize}

\section{Browser page}

Open the browser and go the the link below.

\begin{center}
\url{http://localhost:8080}.
\end{center}

Mind changing the port, if you have modified it.

\section{Stopping the container}

To stop the containers, use

\begin{center}
\verb|docker compose down| \\
\verb|docker-compose down|
\end{center}

depending on your installation.

\subsection{Resources cleaning}

To free up space and clean up unused Docker resources, you can run the following command, which will delete all unused containers, images, volumes, and networks.

\begin{center}
\verb|docker system prune -a|
\end{center}

\subsection{Restart application}

To start the server applications once more, simply repeat the commands found at point 3.
