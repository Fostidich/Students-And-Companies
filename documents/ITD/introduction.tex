\chapter{Introduction}

This document outlines the implementation and testing procedures that have been followed in order to develop a functioning prototype of the service described in the ”Requirements Analysis and Specification Document” and ”Design Document”.

\section{Definitions, acronyms, abbreviations}

\subsection{Definitions}

\begin{itemize}
    \item \textbf{Internship project}: the description of the skills, technologies and roles the student will be working with during the internship, along with the set of tasks that will be covered
    \item \textbf{Internship advertisement}: the public post created by companies to promote available internships on the platform, aimed at attracting suitable candidates by highlighting its key aspects
    \item \textbf{Internship information}: general data about the (ongoing) internship, including the elapsed and remaining time, the compensation and the description of the project the student is working on
    \item \textbf{Enrollment request}: the submission of a student to indicate interest in a specific internship, initiating the selection process by formally applying
    \item \textbf{Enrollment suggestion}: the recommendation made by the platform to guide students in finding projects that best suit them
    \item \textbf{Custom questionnaire}: the tailored set of questions used by companies during interviews to assess a candidate fit for the internship
    \item \textbf{Candidate student}: a student who has applied for an internship and is currently under consideration by a company, moving forward in the selection process
    \item \textbf{Eligible student}: a student who meets the qualifications for an internship, making them viable candidates for recommendation and application
    \item \textbf{Suitable student}: a student who meets the qualifications for an internship, making them potential candidates to be recommended in the companies feed
    \item \textbf{Complaint}: a report submitted by a student or company to the university, regarding issues during the internship, such as unmet expectations, mistreatments, or procedural problems
    \item \textbf{Feedback form}: a structured form for students and companies used to provide feedback on their internship experience, enabling the platform to gather data for analysis, improvements, and recommendations
\end{itemize}

\subsection{Acronyms}

\begin{itemize}
    \item \textbf{S\&C}: Students\&Companies
\end{itemize}

\subsection{Abbreviations}

\begin{itemize}
    \item \textbf{Rn}: n-th requirement
    \item \textbf{KFn}: n-th key function
\end{itemize}

\section{Revision history}

\begin{itemize}
    \item \textbf{Revised on}: \today
    \item \textbf{Version}: 1.0
    \item \textbf{Description}: document initial release
\end{itemize}

\section{Reference documents}

\begin{itemize}
    \item \textbf{Polimi Software Engineering 2 AY 2024/2025 assignment document}: goal, schedule and rules of the requirement engineering and design project
    \item \textbf{Polimi Software Engineering 2 AY 2024/2025 course slides}: the lecture slides provided during the course
\end{itemize}

\section{Document structure}

\begin{itemize}
    \item \textbf{Chapter 1}: this section presents the problem statement and outlines the system objectives. Here are provided the essential initial resources for the readers, to facilitate a comprehensive understanding of the document.
    \item \textbf{Chapter 2}: here are presented all the system functions which the prototype implements.
    \item \textbf{Chapter 3}: this section introduces the programming languages and frameworks that have been chosen, providing comprehensive justifications for each selection.
    \item \textbf{Chapter 4}: this section presents the structure of the code.
    \item \textbf{Chapter 5}: here is provided the information regarding the testing process, specifically outlining the functions that have undergone testing and elucidating the methodology employed.
    \item \textbf{Chapter 6}: this section offer instructions that comprehensively guide users on how to install and run the prototype.
    \item \textbf{Chapter 7}: here is found an estimation of the effort spent by each group member.
    \item \textbf{Chapter 8}: here is provided a list of the references used in this document.
\end{itemize}

